\documentclass{homework}
% \author{Tashfeen, Ahmad}
\class{CSCI 2114: Tashfeen's Data Structures}
% \date{\today}
\title{Extra Credit: Workflow}
\address{%
  Oklahoma City University, %
  Petree College of Arts \& Sciences, %
  Computer Science%
}

\acmfonts
\newcommand\callit[1]{Store the code in a file called \texttt{#1}.}
\newcommand\englishwords{%
  \href{%
    https://raw.githubusercontent.com/dwyl/english-words/master/words_alpha.txt%
  }{370,105 unique English words}%
}

\begin{document} \maketitle

\question Download the \englishwords{} and write a Java program that prints
all the palindromes\footnote{Words that read the same forwards and
  backwards \eg madam.} therein that are more than three letters
long. Listing \ref{pl} shows how to fill a Java array of the
appropriate size with the lines of a file.
\callit{Palindrome.java}

\lstinputlisting[
  linerange={19-28},
  language={java},
  caption={Java function to fill an array with a file.},
  label=pl]
{code/Palindrome.java}

\question How many three or more letter palindromes did your program find?
Which is your favourite one?

\section{Example Executions}

Figure \ref{exmp} shows how the output of the code for the files
\texttt{Palindrome.java} should look like on the standard out. All
your programs must compile/run from the command line using
\texttt{javac} and \texttt{java} commands, e. g.,

\begin{verbatim}
javac Program.java
java Program
\end{verbatim}

\img<exmp>[0.5]
{Example execution of the code for the first question.}{media/example.png}

\section{Submission Instructions}

\begin{itemize}
  \item Submit the source file \texttt{Palindrome.java}. We don't need any
        dot class files.
  \item The PDF file \texttt{sol.pdf} should contain written answers to
        questions as well as a screenshot similar to the one in figure \ref{exmp}
        that demonstrates your code being compiled and ran.
\end{itemize}

\end{document}
