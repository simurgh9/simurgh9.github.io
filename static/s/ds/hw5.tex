\documentclass{homework}
% \author{Tashfeen, Ahmad}
\class{CSCI 2114: Tashfeen's Data Structures}
% \date{\today}
\title{Homework 5}
\address{%
  Oklahoma City University, %
  Petree College of Arts \& Sciences, %
  Computer Science%
}

\acmfonts
\graphicspath{{./media}}

\begin{document} \maketitle

\begin{minipage}[t]{0.45\textwidth}
  \begin{em}
    Est genus in totidem tenui ratione redactum

    scriptula, quot menses lubricus annus habet;

    Parua tabella capit ternos utrimque lapillos,

    in qua uicisse est continuasse suos.
  \end{em}
\end{minipage}
\begin{minipage}[t]{0.5\textwidth}
  There is another game divided into as many parts

  as there are months in the year;

  A small board has three pieces on either side,

  the winner must get all the pieces in a straight line.
\end{minipage}
\begin{flushright}
  ---Ovid (Ars Amatoria III, lines 365-369)
\end{flushright}

\question Read Chapter 5, Sections 5.1 and 5.2 (pg. 161--167) of
``Stuart Russell and Peter Norvig. \emph{Artificial Intelligence A
  Modern Approach} $3^\text{rd}$ edition. (2010).'' There is a copy at
the library. No--seriously, \emph{read ith}!

\question Implement the Minimax algorithm for adversarial search in the
\href{https://en.wikipedia.org/wiki/Tic-tac-toe}{Tic-tac-toe} game tree.

Start by \href{https://tashfeen.org/s/ds/minimax.zip}{downloading}
and reading the starter code. You need to only edit the
\texttt{Minimax.java} by implementing the unfinished Java
methods. Upon successfully implementing all of the API, you may run
\texttt{javac Game.java} and \texttt{java Game}. You'll be presented
to make the first move.  Click at the appropriate place to start the
game.

\img<example>[0.15]{%
  Gameplay in the state of the art graphics by Tashfeen Studios$^\text{TM}$.
}{one, two, three, four, five}

Run \texttt{Test.java} and see if all the tests pass. It verifies the
mathematical properties of the tree model you wrote.

\question Once your game is functional and your \texttt{Minimax.java} class
contains a game tree.  You should start another Java class like this,
\begin{lstlisting}[language=Java]
public class Investigate {
    public static void main(String[] args) {
        Minimax model = new Minimax(3);
        System.out.println(model.root);
    }
}
\end{lstlisting}

Write appropriate code in this file to answer the following questions,

\begin{enumerate}
  \item How many tree-leaves result in a draw?
  \item How many of these leaves  win for the first (max) player?
  \item How many of these leaves win for the second (min) player?
\end{enumerate}

\question Ask two non-CS professors to play your implementation on
your computer.  Before they play, ask them if they think they can win
and and after that how do they think the ``AI'' maybe picking its
moves.  State which professors (and their department) you interviewed
and summarise their responses.

\section*{Submission Instructions}

\begin{enumerate}
  \item Turn in a PDF containing any plots, figures and/or answers from the homework.
  \item Turn in your, \texttt{Minimax.java} and \texttt{Investigate.java}.
\end{enumerate}

\end{document}
